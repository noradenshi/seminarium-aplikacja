\chapter{Opis technologii i algorytmów}
\label{cha:opisTechnologiiIAlgorytmow}

Niniejszy rozdział przedstawia technologie oraz algorytmy wykorzystane w opracowanym systemie do losowania zestawów zadań na kolokwium. Omówione zostaną zarówno narzędzia programistyczne, jak i logika działania aplikacji.

%---------------------------------------------------------------------------

\section{Technologie użyte w systemie}
\label{sec:technologieUzyteWSystemie}

System został zaimplementowany jako aplikacja webowa w języku PHP, co umożliwia łatwe uruchamianie w przeglądarce internetowej bez konieczności instalacji dodatkowego oprogramowania. Wybrane technologie i narzędzia to:

\begin{enumerate}
      \item PHP – język skryptowy po stronie serwera, wykorzystywany do logiki biznesowej, zarządzania sesjami i generowania dynamicznego HTML.
      \item HTML i CSS – służą do tworzenia interfejsu użytkownika oraz stylizacji strony, w tym przycisków, formularzy i obszarów wyświetlania historii losowań.
      \item Mechanizm sesji w PHP \texttt{ session\_start() } – pozwala na przechowywanie danych między kolejnymi wywołaniami strony, co umożliwia zachowanie stanu aplikacji, np. listy pozostałych zadań i historii losowań.
      \item Przeglądarka internetowa – umożliwia interakcję użytkownika z aplikacją oraz wyświetlanie wyników losowań w formie czytelnej dla nauczyciela lub administratora.
\end{enumerate}

Zastosowanie tych technologii pozwala na prostą, a jednocześnie funkcjonalną implementację systemu, który jest dostępny w dowolnym środowisku z obsługą PHP i przeglądarki internetowej.

%---------------------------------------------------------------------------

\section{Struktura danych}
\label{sec:strukturaDanych}

Podstawowym elementem systemu jest tablica wielowymiarowa \lstinline|$tasks|, która przechowuje wszystkie dostępne zadania wraz z ich typami. Każdy element tablicy zawiera:

\begin{itemize}
      \item nazwę zadania (ciąg tekstowy),
      \item typy zadań (np. EX – Excel, SO – Excel, BS, DZ – Java).
\end{itemize}

Podczas inicjalizacji aplikacji tablica \lstinline|$tasks| jest przetwarzana na dwie listy: \lstinline|$tasksExcel| i \lstinline|$tasksJava|, które przechowują zadania dostępne do losowania dla każdej kategorii.
Ponadto w sesji PHP przechowywane są:

\begin{itemize}
      \item numer kolejnego studenta \lstinline|$_SESSION['student']|,
      \item historia losowań \lstinline|$_SESSION['records']|,
      \item aktualne listy zadań do wylosowania \lstinline|$_SESSION['tasksExcel']| i \lstinline|$_SESSION['tasksJava']|.
\end{itemize}

Taka struktura danych pozwala na szybkie losowanie zadań, uniknięcie powtarzalności oraz łatwe wyświetlanie wyników w interfejsie użytkownika.

%---------------------------------------------------------------------------

\section{Algorytm Losowania Zestawow}
\label{sec:algorytmLosowaniaZestawow}

Proces losowania opiera się na kilku krokach:

\begin{enumerate}
      \item Sprawdzenie, czy sesja została zainicjalizowana – jeśli nie, system tworzy listy zadań Excel i Java oraz inicjalizuje licznik studentów i historię losowań.
      \item Losowanie zadania – z list \lstinline|$tasksExcel| i \lstinline|$tasksJava| losowane są po jednym zadaniu dla kolejnego studenta przy użyciu funkcji \lstinline|rand()|.
      \item Usunięcie wylosowanego zadania – \lstinline|array_splice()| usuwa wybrane zadania z list, aby uniknąć powtórzeń w przyszłych losowaniach.
      \item Rejestracja wyniku losowania – wylosowany zestaw zapisywany jest w tablicy \lstinline|$_SESSION['records']| i numer student jest inkrementowany.
      \item Wyświetlenie wyników – w interfejsie użytkownika prezentowana jest lista wszystkich wylosowanych zestawów.
      \item Algorytm jest prosty, ale skuteczny – zapewnia losowy, unikalny przydział zadań przy minimalnym nakładzie obliczeniowym.
\end{enumerate}


%---------------------------------------------------------------------------

\section{Bezpieczeństwo i ograniczenia algorytmu}
\label{sec:bezpieczenstwoIOgraniczniaAlgorytmu}

Chociaż algorytm losowania działa poprawnie w środowisku lokalnym lub dla jednego użytkownika, istnieją pewne ograniczenia:
\begin{itemize}
      \item Brak obsługi wielu użytkowników jednocześnie – w przypadku równoległego dostępu do systemu, sesje mogą się nadpisywać.
      \item Brak walidacji danych wejściowych – formularz nie zabezpiecza przed niepoprawnym wywołaniem metody POST.
      \item Ograniczona skalowalność – lista zadań przechowywana w pamięci sesji może być niewystarczająca dla dużej liczby studentów lub bardzo rozbudowanego zestawu zadań.
      \item Pomimo tych ograniczeń, system jest wystarczający do zastosowań edukacyjnych w małej skali i stanowi dobrą podstawę do dalszej rozbudowy.
\end{itemize}

%---------------------------------------------------------------------------


