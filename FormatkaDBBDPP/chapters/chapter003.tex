\chapter{Prezentacja wyników}
\label{cha:prezentacjaWynikow}

Celem niniejszego rozdziału jest przedstawienie działania opracowanego systemu, 
jego interfejsu użytkownika oraz efektów losowania zestawów zadań dla studentów. 
W rozdziale pokazano, w jaki sposób aplikacja prezentuje wyniki, w jaki sposób użytkownik wchodzi w interakcję z systemem oraz przykładowe print screeny ilustrujące funkcjonalność aplikacji.

% ------------------------------------------------------------------------

\section{Opis aplikacji}
\label{sec:opisAplikacji}

Aplikacja została zaprojektowana jako prosty system webowy, umożliwiający losowanie zestawów zadań w dwóch kategoriach: Java i Excel. Interfejs użytkownika składa się z następujących elementów:
\begin{itemize}
    \item Nagłówek strony – zawiera nazwę systemu „IRZ” i pełni funkcję informacyjną.
    \item Obszar historii losowań – wyświetla wylosowane zestawy dla kolejnych studentów w postaci listy tekstowej. Każdy wpis zawiera numer studenta oraz przydzielone zadania w obu kategoriach.
    \item Przycisk losowania zestawu – po kliknięciu generuje nowy zestaw zadań dla kolejnego studenta. Przycisk jest dezaktywowany, jeśli nie ma już dostępnych zadań w którejkolwiek kategorii.
    \item Aplikacja wykorzystuje sesje PHP do przechowywania stanu między kolejnymi wywołaniami strony, dzięki czemu wszystkie wylosowane zestawy pozostają widoczne w historii, nawet po odświeżeniu strony.
\end{itemize}

% ------------------------------------------------------------------------

\section{Prezentacja wyników losowania}
\label{sec:prezentacjaWynikowLosowania}

Po naciśnięciu przycisku „Losuj zestaw”, aplikacja wykonuje następujące kroki:
\begin{enumerate}
    \item Losuje po jednym zadaniu z list tasksExcel i tasksJava.
    \item Tworzy rekord w formacie:
    \begin{lstlisting}[language=PHP]
        Osoba: [numer studenta]
        [zadanie Java]
        [zadanie Excel]
    \end{lstlisting}
    \item Dodaje rekord do historii \lstinline|$_SESSION['records']| i wyświetla go w obszarze wyników.
    \item Aktualizuje stan list zadań, usuwając już wylosowane elementy, aby zapewnić unikalność zestawów.
\end{enumerate}

Dzięki temu każdy student otrzymuje unikalny zestaw zadań, a nauczyciel może w prosty sposób śledzić przebieg losowań oraz sprawdzić, jakie zadania zostały już przydzielone.

% ------------------------------------------------------------------------

\section{Intefrejs użytkownika}
\label{sec:interfejsUzytkownika}

Interfejs użytkownika został zaprojektowany z myślą o czytelności i prostocie obsługi. Główne cechy:
\begin{itemize}
    \item Minimalistyczny design – wszystkie elementy rozmieszczone w centralnym oknie o stałej szerokości i wysokości.
    \item Wyraźne oznaczenie kategorii zadań – w nagłówku okna podawana jest liczba pozostałych zadań w każdej kategorii.
    \item Historia losowań w formie listy – każde losowanie jest wyświetlane w osobnym polu <p> z wyróżnionym numerem studenta.
    \item Przycisk aktywowany tylko przy dostępnych zadaniach – uniemożliwia losowanie, jeśli któraś z kategorii jest już pusta.
\end{itemize}

% ------------------------------------------------------------------------

\section{Print screeny aplikacji}
\label{sec:printScreenyAplikacji}

W celu lepszego zobrazowania działania systemu, poniżej przedstawiono przykładowe zrzuty ekranu:
\begin{enumerate}
    \item Ekran początkowy aplikacji – pokazuje nagłówek, liczby pozostałych zadań w każdej kategorii oraz przycisk losowania.
            [Tutaj dodać zdj]
    \item Historia losowań po kilku losowaniach – wyświetlone zostały wylosowane zestawy dla kilku studentów.
            [Tutaj dodać zdj]
    \item Przycisk losowania dezaktywowany – widok po wyczerpaniu list zadań, pokazujący, że system uniemożliwia dalsze losowania.
            [Tutaj dodać zdj]
\end{enumerate}

Print screeny umożliwiają wizualne przedstawienie działania aplikacji, potwierdzając poprawność implementacji algorytmu losowania i funkcjonalności interfejsu użytkownika.

% ------------------------------------------------------------------------


