\chapter{Prezentacja wyników}
\label{cha:prezentacjaWynikow}

Celem niniejszego rozdziału jest przedstawienie działania opracowanego systemu, 
jego interfejsu użytkownika oraz efektów losowania zestawów zadań dla studentów. 
W rozdziale pokazano, w jaki sposób aplikacja prezentuje wyniki, w jaki sposób użytkownik wchodzi w interakcję z systemem oraz przykładowe print screeny ilustrujące funkcjonalność aplikacji.

% ------------------------------------------------------------------------

\section{Opis aplikacji}
\label{sec:opisAplikacji}

Aplikacja została zaprojektowana jako prosty system webowy, umożliwiający losowanie zestawów zadań w dwóch kategoriach: Java i Excel. Interfejs użytkownika składa się z następujących elementów:
\begin{itemize}
    \item Nagłówek strony – zawiera nazwę systemu „IRZ” i pełni funkcję informacyjną.
    \item Obszar historii losowań – wyświetla wylosowane zestawy dla kolejnych studentów w postaci listy tekstowej. Każdy wpis zawiera numer studenta oraz przydzielone zadania w obu kategoriach.
    \item Przycisk losowania zestawu – po kliknięciu generuje nowy zestaw zadań dla kolejnego studenta. Przycisk jest dezaktywowany, jeśli nie ma już dostępnych zadań w którejkolwiek kategorii.
    \item Aplikacja wykorzystuje sesje PHP do przechowywania stanu między kolejnymi wywołaniami strony, dzięki czemu wszystkie wylosowane zestawy pozostają widoczne w historii, nawet po odświeżeniu strony.
\end{itemize}

% ------------------------------------------------------------------------

\section{Prezentacja wyników losowania}
\label{sec:prezentacjaWynikowLosowania}

Po naciśnięciu przycisku „Losuj zestaw”, aplikacja wykonuje następujące kroki:
\begin{enumerate}
    \item Losuje po jednym zadaniu z list tasksExcel i tasksJava.
    \item Tworzy rekord w formacie:
    \begin{lstlisting}[language=PHP]
        Osoba: [numer studenta]
        [zadanie Java]
        [zadanie Excel]
    \end{lstlisting}
    \item Dodaje rekord do historii \lstinline|$_SESSION['records']| i wyświetla go w obszarze wyników.
    \item Aktualizuje stan list zadań, usuwając już wylosowane elementy, aby zapewnić unikalność zestawów.
\end{enumerate}

Dzięki temu każdy student otrzymuje unikalny zestaw zadań, a nauczyciel może w prosty sposób śledzić przebieg losowań oraz sprawdzić, jakie zadania zostały już przydzielone.

% ------------------------------------------------------------------------

\section{Intefrejs użytkownika}
\label{sec:interfejsUzytkownika}

Interfejs użytkownika został zaprojektowany z myślą o czytelności i prostocie obsługi. Główne cechy:
\begin{itemize}
    \item Minimalistyczny design – wszystkie elementy rozmieszczone w centralnym oknie o stałej szerokości i wysokości.
    \item Wyraźne oznaczenie kategorii zadań – w nagłówku okna podawana jest liczba pozostałych zadań w każdej kategorii.
    \item Historia losowań w formie listy – każde losowanie jest wyświetlane w osobnym polu <p> z wyróżnionym numerem studenta.
    \item Przycisk aktywowany tylko przy dostępnych zadaniach – uniemożliwia losowanie, jeśli któraś z kategorii jest już pusta.
\end{itemize}

% ------------------------------------------------------------------------

\section{Print screeny aplikacji}
\label{sec:printScreenyAplikacji}

W celu lepszego zobrazowania działania systemu, poniżej przedstawiono przykładowe zrzuty ekranu:
\begin{enumerate}
    \item Ekran początkowy aplikacji – pokazuje nagłówek, liczby pozostałych zadań w każdej kategorii oraz przycisk losowania.
            [Tutaj dodać zdj]
    \item Historia losowań po kilku losowaniach – wyświetlone zostały wylosowane zestawy dla kilku studentów.
            [Tutaj dodać zdj]
    \item Przycisk losowania dezaktywowany – widok po wyczerpaniu list zadań, pokazujący, że system uniemożliwia dalsze losowania.
            [Tutaj dodać zdj]
\end{enumerate}

Print screeny umożliwiają wizualne przedstawienie działania aplikacji, potwierdzając poprawność implementacji algorytmu losowania i funkcjonalności interfejsu użytkownika.

% ------------------------------------------------------------------------

\section{Wnioski}
\label{sec:wnioski}

Na podstawie przeprowadzonych testów oraz analizy działania aplikacji można stwierdzić, że system spełnia swoje podstawowe założenia funkcjonalne. 
Aplikacja poprawnie losuje zestawy zadań z dwóch kategorii, zapewniając ich unikalność oraz umożliwiając użytkownikowi śledzenie pełnej historii przydzielonych zestawów. 
Interfejs użytkownika jest czytelny i intuicyjny, co pozwala na wygodne korzystanie z systemu bez konieczności posiadania wiedzy technicznej.

Ważnym elementem implementacji jest wykorzystanie mechanizmu sesji, który umożliwia przechowywanie bieżącego stanu aplikacji, 
w tym numeru studenta oraz list pozostałych zadań. Rozwiązanie to okazuje się wystarczające w środowisku jednostanowiskowym i przy niewielkiej liczbie użytkowników.

Pomimo swojej prostoty system pokazuje, że nawet niewielkie narzędzie webowe może skutecznie wspierać proces dydaktyczny, automatyzując czynności, 
które tradycyjnie wykonywano ręcznie. Eliminacja powtarzalnych zadań oraz ułatwiona kontrola nad przebiegiem losowania to główne zalety opracowanego rozwiązania.

% ------------------------------------------------------------------------

\section{Możliwości rozbudowy systemu}
\label{sec:mozliwosciRozbudowySystemu}

Aplikacja, pomimo poprawnego działania w obecnej formie, może zostać rozszerzona o dodatkowe funkcjonalności, które zwiększyłyby jej użyteczność oraz skalowalność. Do najważniejszych możliwości rozbudowy należą:

\begin{enumerate}
    \item Zapis wyników do bazy danych
            Zamiast przechowywania wyników w sesji, możliwe byłoby zapisanie historii losowań w relacyjnej bazie danych (np. MySQL). Pozwoliłoby to na trwałe archiwizowanie danych, analizę wyników, filtrowanie i generowanie raportów.
    \item Obsługa wielu użytkowników
            Aktualna wersja zakłada pracę jednego użytkownika na jednej instancji sesji. Rozbudowa o system logowania lub identyfikacji nauczycieli umożliwiłaby korzystanie z aplikacji w środowisku wielodostępnym.
    \item Dynamiczne zarządzanie listą zadań 
        System można rozbudować o formularz umożliwiający dodawanie nowych zadań, usuwanie istniejących lub zmianę kategorii zadań bez konieczności edycji kodu źródłowego.
    \item Eksport wyników
        Możliwość eksportu historii losowań do pliku CSV lub PDF ułatwiłaby dokumentowanie pracy nauczyciela oraz przechowywanie wyników w innych systemach.
    \item Rozbudowa interfejsu użytkownika
        Można zastosować framework CSS (np. Bootstrap) lub bibliotekę JavaScript, aby zwiększyć ergonomię, dodać animacje, licznik pozostałych zadań czy bardziej rozbudowane komunikaty dla użytkownika.
    \item Wsparcie dla różnych grup i przedmiotów
        System można rozbudować tak, aby losowane zadania nie powtarzały się między różnymi grupami studentów, jednocześnie dopuszczając powtarzalność, 
        jeśli studenci piszą w tym samym dniu lub tej samej godzinie. Dodatkowo możliwe jest wprowadzenie obsługi różnych przedmiotów,
        dla których tworzona byłaby osobna pula zadań, co zwiększyłoby elastyczność systemu i umożliwiłoby stosowanie go w różnych kursach.
\end{enumerate}

Wprowadzenie powyższych usprawnień mogłoby znacząco zwiększyć możliwości aplikacji i dostosować ją do bardziej zaawansowanych lub komercyjnych potrzeb.